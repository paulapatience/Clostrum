\chapter{Evaluation environment}

\section{Mixin class}

\Defclass {evaluation-environment-mixin}

This class should be used in order to create an evaluation environment
class that has this mixin class and a run-time environment class as
superclasses.

\Definitarg {:parent}

This initialization argument must be provided and should be either
another evaluation environment, or a run-time-environment.

\Defgeneric {parent} {environment}

\Defmethod {parent} {(environment {\tt evaluation-environment-mixin})}

This generic function returns the environment objection that was passed
as the \texttt{:parent} initialization argument when
\textit{environment} was created.

\section{Methods}

{\small\Defmethod {fboundp} {client (environment {\tt
      evaluation-environment-mixin}) \\ function-name}
}

This method first calls \texttt{(call-next-method)}.  If that call
returns true, then that true value is returned.  If that call returns
false, then this method calls \texttt{fboundp} with the same
\texttt{client} argument, a second argument resulting from calling
\texttt{(parent environment)} and the same \texttt{function-name}
argument.

{\small\Defmethod {special-operator} {client (environment {\tt
      evaluation-environment-mixin}) \\ function-name}
}

This method first calls \texttt{(call-next-method)}.  If that call
returns true, then that true value is returned.  If that call returns
false, then this method calls \texttt{special-operator} with the same
\texttt{client} argument, a second argument resulting from calling
\texttt{(parent environment)} and the same \texttt{function-name}
argument.

{\small\Defmethod {fdefinition} {client (environment {\tt
      evaluation-environment-mixin}) \\ function-name}
}

This method first calls \texttt{(call-next-method)}.  If that call
returns true, then that true value is returned.  If that call returns
false, then this method calls \texttt{fdefinition} with the same
\texttt{client} argument, a second argument resulting from calling
\texttt{(parent environment)} and the same \texttt{function-name}
argument.

{\small\Defmethod {macro-function} {client (environment {\tt
      evaluation-environment-mixin}) \\ symbol}
}

This method first calls \texttt{(call-next-method)}.  If that call
returns true, then that true value is returned.  If that call returns
false, then this method calls \texttt{macro-function} with the same
\texttt{client} argument, a second argument resulting from calling
\texttt{(parent environment)} and the same \texttt{symbol}
argument.

{\small\Defmethod {compiler-macro-function} {client (environment {\tt
      evaluation-environment-mixin}) \\ function-name}
}

This method first calls \texttt{(call-next-method)}.  If that call
returns true, then that true value is returned.  If that call returns
false, then this method calls \texttt{compiler-macro-function} with
the same \texttt{client} argument, a second argument resulting from
calling \texttt{(parent environment)} and the same
\texttt{function-name} argument.

{\small\Defmethod {function-type} {client (environment {\tt
      evaluation-environment-mixin}) \\ function-name}
}

This method first calls \texttt{(call-next-method)}.  If that call
returns true, then that true value is returned.  If that call returns
false, then this method calls \texttt{function-type} with the same
\texttt{client} argument, a second argument resulting from calling
\texttt{(parent environment)} and the same \texttt{function-name}
argument.

MORE METHODS HERE...
