\chapter{Evaluation environment}

\section{Mixin class}

\Defclass {evaluation-environment-mixin}

This class should be used in order to create an evaluation environment
class that has this mixin class and a run-time environment class as
superclasses \emph{in that order}.  It is important that this mixin
class precede the run-time environment class in the list of
superclasses, so that the methods described in this chapter
(specialized to \texttt{evaluation-environment-mixin}) are more
specific than the methods specialized to
\texttt{run-time-environment}.

\Definitarg {:parent}

This initialization argument must be provided and should be either
another evaluation environment, or a run-time environment.  In most
cases, it should be a run-time environment, i.e., an instance of the
class \texttt{run-time-environment}.

\Defgeneric {parent} {environment}

\Defmethod {parent} {(environment {\tt evaluation-environment-mixin})}

This generic function returns the environment object that was passed
as the \texttt{:parent} initialization argument when
\textit{environment} was created.

\section{Methods}

{\small\Defmethod {special-operator} {client (environment {\tt
      evaluation-environment-mixin}) \\ function-name}
}

This method first executes \texttt{(call-next-method)}.  If the return
value of that call is non-\texttt{nil}, it is returned.  If the return
value of the call \texttt{(call-next-method)} is \texttt{nil}, then
this method calls \texttt{special-operator} with the same
\textit{client} argument, a second argument resulting from calling
\texttt{(parent} \textit{environment}\texttt{)} and the same
\textit{function-name} argument, and it returns the value of that
call.

In a typical system, there are no provisions for adding new special
operators at compile time to be used only at compile time, so the call
to \texttt{(call-next-method)} would then return \texttt{nil}.  But we
see no reason to disallow the possibility of adding new special
operators at compile time in case some clients need such a feature.

{\small\Defmethod {fdefinition} {client (environment {\tt
      evaluation-environment-mixin}) \\ function-name}
}

This method calls \texttt{fdefinition} with the same \textit{client}
argument, a second argument resulting from calling \texttt{(parent}
\textit{environment}\texttt{)} and the same \textit{function-name}
argument.

{\small\Defmethod {macro-function} {client (environment {\tt
      evaluation-environment-mixin}) \\ symbol}
}

This method calls \texttt{macro-function} with the same
\textit{client} argument, a second argument resulting from calling
\texttt{(parent} \textit{environment}\texttt{)} and the same
\textit{symbol} argument.

{\small\Defmethod {compiler-macro-function} {client (environment {\tt
      evaluation-environment-mixin}) \\ function-name}
}

This method calls \texttt{compiler-macro-function} with the same
\textit{client} argument, a second argument resulting from calling
\texttt{(parent} \textit{environment}\texttt{)} and the same
  \textit{function-name} argument.

{\small\Defmethod {function-type} {client (environment {\tt
      evaluation-environment-mixin}) \\ function-name}
}

This method calls \texttt{function-type} with the same \textit{client}
argument, a second argument resulting from calling \texttt{(parent}
\textit{environment}\texttt{)} and the same \textit{function-name}
argument.

MORE METHODS HERE...
