\chapter{General principles}

\sysname{} is designed to provide most of the functionality that
typical clients will need.  But \sysname{} is also designed to be
extended by clients that need additional functionality.

The main mechanism for providing extensions is the use of a
\texttt{client} parameter in all of the generic functions defined by
\sysname{}.  None of the methods of the generic functions defined by
\sysname{} specialize to the \texttt{client} parameter.  However, it is
highly recommended that client code respect these two conventions:

\begin{enumerate}
\item Clients calling a \sysname{} generic function should always
  supply an instance of some client-specific standard object as the
  \texttt{client} argument.
\item Client-specific methods on \sysname{} generic functions should
  always specialize to some client-specific class.
\end{enumerate}

These rules exist so that several different clients may coexist in
the same \commonlisp{} image, without interfering with one another.
