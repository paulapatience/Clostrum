\chapter{Compilation environment}

\section{Class}

\Defclass {compilation-environment}

An instance of this class is passed as the \texttt{\&environment}
argument to macro functions during compile time.

\Definitarg {:parent}

This initialization argument must be provided and should be
an evaluation environment.

\Defgeneric {parent} {environment}

\Defmethod {parent} {(environment {\tt compilation-environment})}

This generic function returns the environment objection that was passed
as the \texttt{:parent} initialization argument when
\textit{environment} was created.

\section{Generic functions}

{\small\Defgeneric {(setf function-description)} {description client environment \\
function-name}
}

This generic function is typically called by client code in the
expansion of one of the macros \texttt{defun} and
\texttt{defgeneric} in the \texttt{:compile-toplevel} situation of an
\texttt{eval-when} form.  The \textit{description} argument is then an
object that describes the function being defined.  It would typically
contain the following information:

\begin{itemize}
\item The lambda list of the function.
\item The class name of the function.  For an ordinary function, this
  name may be just \texttt{function}, and for a generic function, it
  is the argument of the \texttt{:generic-function-class}
  \texttt{defgeneric} option.
\item The class name for methods if it is a generic function.  It is
  the argument of the \texttt{:method-class} \texttt{defgeneric}
  option.
\item The method combination if it is a generic function.  It is the
  argument of the \texttt{:method-combination} \texttt{defgeneric}
  option.
\item Information for possible inlining.
\item Information about an associated compiler macro.
\item The function type.
\item Information about source location.
\end{itemize}

A value of \texttt{nil} for the \textit{description} argument deletes
the existing information associated with \textit{function-name} in
\textit{environment}.

Before calling this generic function, client code might want to verify
that a call to \texttt{macro-function} returns \texttt{nil}.  If not,
it means that \textit{function-name} already names a macro, either in
the evaluation environment or in the run-time environment.

\Defgeneric {function-description} {client environment \\ function-name}

This generic function returns the function description provided by the
generic function \texttt{(setf function-description)}.  If no function
description has been provided, then \texttt{nil} is returned.

{\small\Defgeneric {(setf variable-description)} {description client environment \\
symbol}
}

This generic function is typically called by client code in the
expansion of one of the macros \texttt{devar} and
\texttt{defparameter} in the \texttt{:compile-toplevel} situation of an
\texttt{eval-when} form.  The \textit{description} argument is then an
object that describes the variable being defined.  It would typically
contain the following information:

\begin{itemize}
\item The variable type.
\item Information about source location.
\end{itemize}

A value of \texttt{nil} for the \textit{description} argument, deletes
the existing information associated with \textit{symbol} in
\textit{environment}.

\Defgeneric {variable-description} {client environment \\ symbol}

This generic function returns the function description provided by the
generic function \texttt{(setf variable-description)}.  If no variable
description has been provided, then \texttt{nil} is returned.

{\small\Defgeneric {(setf class-description)} {description client environment \\
symbol}
}

This generic function is typically called by client code in th
expansion of the \texttt{defclass} macro in the
\texttt{:compile-toplevel} situation of an \texttt{eval-when} form.
The \textit{description} argument is then an object that describes the
class being defined.  It would typically contain the following
information:

\begin{itemize}
\item The name of the class.
\item The name of its superclasses.
\item The name of its metaclass.
\item Information about source location.
\end{itemize}

A value of \texttt{nil} for the \textit{description} argument, deletes
the existing information associated with \textit{symbol} in
\textit{environment}.

\Defgeneric {class-description} {client environment \\ symbol}

This generic function returns the function description provided by the
generic function \texttt{(setf class-description)}.  If no variable
description has been provided, then \texttt{nil} is returned.

MORE GENERIC FUNCTIONS HERE

\section{Methods}

{\small\Defmethod {macro-function} {client (environment {\tt
      compilation-environment-mixin}) \\ symbol}
}

This method calls \texttt{macro-function} with the same
\textit{client} argument, with the value of calling \texttt{parent} on
\textit{environment}, and with the same \textit{symbol} argument.

{\small\Defmethod {(setf macro-function)} {function client (environment {\tt
      compilation-environment-mixin}) \\ symbol}
}

This method calls \texttt{(setf macro-function)} with the same
\textit{function} argument, with the same \textit{client} argument, with
the value of calling \texttt{parent} on \textit{environment}, and with
the same \textit{symbol} argument.

Before client code calls this generic function, it might want to
verify that a call to \texttt{function-description} returns
\texttt{nil}, and take action if this is not the case, because then it
means that a definition of a function and a definition of a macro of
the same name both exist in the same compilation unit.  Possible
actions include calling \texttt{(setf function-description)} with a
description of \texttt{nil}, thereby effectively replacing the
function by the macro.
