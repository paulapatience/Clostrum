\chapter{Compilation environment}

\section{Class}

\Defclass {compilation-environment}

An instance of this class is passed as the \texttt{\&environment}
argument to macro functions during compile time.

\Definitarg {:parent}

This initialization argument must be provided and should be
an evaluation environment.

\Defgeneric {parent} {environment}

\Defmethod {parent} {(environment {\tt compilation-environment})}

This generic function return the environment objection that was passed
as the \texttt{:parent} initialization argument when
\textit{environment} was created.

\section{Generic functions}

\Defgeneric {function-class-name} {client environment \\ function-name}

This generic function returns the name of a the class of the function
named \textit{function-name}.  If no information about a generic
function with that name exists in \textit{environment}, then
\texttt{nil} is returned.

\Defgeneric {(setf function-class-name)} {class-name client environment \\
function-name}

This generic function sets the name of a the class of the function
named \textit{function-name}.

Information about the class of a function is useful mostly for generic
functions, but can also be used for ordinary functions and instances
of the class named \texttt{funcallable-standard-object}, defined by
the metaobject protocol.

\Defgeneric {method-class-name} {client environment \\ function-name}

This generic function returns the name of a the class of the methods
on the generic function named \textit{function-name}.  If no information
about a generic function with that name exists in
\textit{environment}, then \texttt{nil} is returned.

\Defgeneric {(setf method-class-name)} {class-name client environment \\
function-name}

This generic function sets the name of a the class of the methods on
the generic function named \textit{function-name}.
