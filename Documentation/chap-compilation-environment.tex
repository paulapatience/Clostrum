\chapter{Compilation environment}

\section{Class}

\Defclass {compilation-environment}

An instance of this class is passed as the \texttt{\&environment}
argument to macro functions during compile time.

\Definitarg {:parent}

This initialization argument must be provided and should be
an evaluation environment.

\Defgeneric {parent} {environment}

\Defmethod {parent} {(environment {\tt compilation-environment})}

This generic function returns the environment objection that was passed
as the \texttt{:parent} initialization argument when
\textit{environment} was created.

\section{Generic functions}

\Defgeneric {(setf function-description)} {description client environment \\
function-name}

This generic function is typically called by client code in the
expansion of one of the macros \texttt{defun} and
\texttt{defgeneric} in the \texttt{:compile-toplevel} situation of an
\texttt{eval-when} form.  The \textit{description} argument is then an
object that describes the function being defined.  It would typically
contain the following information:

\begin{itemize}
\item The lambda list of the function.
\item The class name of the function.  For an ordinary function, this
  name may be just \texttt{function}, and for a generic function, it
  is the argument of the \texttt{:generic-function-class}
  \texttt{defgeneric} option.
\item The class name for methods if it is a generic function.  It is
  the argument of the \texttt{:method-class} \texttt{defgeneric}
  option.
\item The method combination if it is a generic function.  It is the
  argument of the \texttt{:method-combination} \texttt{defgeneric}
  option.
\item Information for possible inlining.
\item Information about an associated compiler macro.
\item Information about source location.
\end{itemize}

A value of \texttt{nil} for the \textit{description} argument deletes
the existing information associated with \textit{function-name} in
\textit{environment}.

\Defgeneric {function-description} {client environment \\ function-name}

This generic function returns the function description provided by the
generic function \texttt{(setf function-description)}.  If no function
description has been provided, then \texttt{nil} is returned.

MORE GENERIC FUNCTIONS HERE
