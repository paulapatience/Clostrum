\chapter{Run-time environment}

\section{Class}

\Defclass {run-time-environment}

This is the base class for run-time environments.

\section{Protocol functions}
\label{sec-run-time-protocol-functions}

\Defgeneric {fboundp} {client environment function-name}

This generic function is a generic version of the \commonlisp{}
function \texttt{fboundp}.

It returns true if \textit{function-name} has a definition in
\textit{environment} as an ordinary function, a generic function, a
macro, or a special operator.

\Defgeneric {fmakunbound} {client environment function-name}

This generic function is a generic version of the \commonlisp{}
function named \texttt{fmakunbound}.

This function makes \textit{function-name} \emph{unbound} in the
function namespace of \textit{environment}.

If \textit{function-name} already has a definition in
\textit{environment} as an ordinary function, as a generic function,
as a macro, or as a special operator, then that definition is lost.

If \textit{function-name} has a \texttt{setf} expander associated with
it, then that \texttt{setf} expander is lost.

\Defgeneric {special-operator} {client environment function-name}

If \textit{function-name} has a definition as a special operator in
\textit{environment}, then that definition is returned.  The
definition is the object that was used as an argument to \texttt{(setf
  special-operator)}.  The exact nature of this object is not
specified, other than that it can not be \texttt{nil}.  If
\textit{function-name} does not have a definition as a special
operator in \textit{environment}, then \texttt{nil} is returned.

\Defgeneric {(setf special-operator)} {new client environment function-name}

Set the definition of \textit{function-name} to be a special operator.
The exact nature of \textit{new} is not specified, except that a
value of \texttt{nil} means that \textit{function-name} no longer has
a definition as a special operator in \textit{environment}.

If a value other than \texttt{nil} is given for \textit{new}, and
\textit{function-name} already has a definition as an ordinary
function, as a generic function, or as a macro, then an error is
signaled.  As a consequence, if it is desirable for
\textit{function-name} to have a definition both as a special operator
and as a macro, then the definition as a special operator should be
set first.

\Defgeneric {fdefinition} {client environment function-name}

This generic function is a generic version of the \commonlisp{}
function named \texttt{(setf cl:fdefinition)}.

This function returns two values. If \textit{function-name} has a definition
in the function namespace of \textit{environment} (i.e., if \texttt{fboundp}
returns true), then the first value is the object representing the
definition. The second value is the function definition type, that is:
\texttt{cl:function}, \texttt{cl:macro-function} or
\texttt{cl:special}. Otherwise a function which signals
\texttt{undefined-error} and \texttt{cl:undefined-function} are returned.

\Defgeneric {(setf fdefinition)} {new-def client environment function-name}

This generic function is a generic version of the \commonlisp{}
function named \texttt{(setf cl:fdefinition)}.

\textit{new-def} must be an ordinary function or a generic function.
If \textit{function-name} already names a function or a macro, then
the previous definition is lost.  If \textit{function-name} already
names a special operator, then an error is signaled.

If \textit{function-name} is a symbol and it has an associated \texttt{setf}
expander, then that \texttt{setf} expander is preserved.

\Defgeneric {macro-function} {client environment symbol}

This generic function is a generic version of the \commonlisp{}
function named \texttt{cl:macro-function}.

If \textit{symbol} has a definition as a macro in
\textit{environment}, then the corresponding macro expansion function
is returned.

If \textit{symbol} has no definition in the function namespace of
\textit{environment}, or if the definition is not a macro, then this
function returns \texttt{nil}.

\Defgeneric {(setf macro-function)} {new-def client environment symbol}

This generic function is a generic version of the \commonlisp{}
function \texttt{(setf cl:macro-function)}.

\textit{new-def} must be a macro expansion function or \texttt{nil}.
A call to this function then always succeeds.  A value of \texttt{nil}
means that the \textit{symbol} no longer has a macro function
associated with it.  If \textit{symbol} already names a macro or a
function, then the previous definition is lost.  If \textit{symbol}
already names a special operator, that definition is kept.

If \textit{symbol} already names a function, then any proclamation of
the type of that function is lost.  In other words, if at some later
point \textit{symbol} is again defined as a function, its proclaimed
type will be \texttt{t}.

If \textit{symbol} already names a function, then any \texttt{inline} or
\texttt{notinline} proclamation of the type of that function is lost.  In other
words, if at some later point \textit{symbol} is again defined as a
function, its proclaimed inline information will be \texttt{nil}.

If \textit{symbol} has an associated \texttt{setf} expander, then that
\texttt{setf} expander is preserved.

\Defgeneric {compiler-macro-function} {client environment function-name}

This generic function is a generic version of the \commonlisp{}
function named \texttt{cl:compiler-macro-function}.

If \textit{function-name} has a definition as a compiler macro in
\textit{environment}, then the corresponding compiler macro function
is returned.

If \textit{function-name} has no definition as a compiler macro in
\textit{environment}, then this function returns \texttt{nil}.

\Defgeneric {(setf compiler-macro-function)}\\
{new-def client environment function-name}

This generic function is a generic version of the \commonlisp{}
function \texttt{(setf cl:compiler-macro-function)}.

\textit{new-def} can be a compiler macro function or \texttt{nil}.
When it is a compiler macro function, then it establishes
\textit{new-def} as a compiler macro for \textit{function-name} and
any existing definition is lost.  A value of \texttt{nil} means that
\textit{function-name} no longer has a compiler macro associated with
it in \textit{environment}.

\Defgeneric {function-type} {client environment function-name}

This generic function returns the proclaimed type of the function
associated with \textit{function-name} in \textit{environment}.

If \textit{function-name} is not associated with an ordinary function
or a generic function in \textit{environment}, then \texttt{nil} is
returned.

If \textit{function-name} is associated with an ordinary function or a
generic function in \textit{environment}, but no type proclamation for
that function has been made, then this generic function returns
\texttt{t}.

\Defgeneric {(setf function-type)} {new-type client environment function-name}

This generic function is used to set the proclaimed type of the
function associated with \textit{function-name} in
\textit{environment} to \textit{new-type}.

If \textit{function-name} is associated with a macro or a special
operator in \textit{environment}, then an error is signaled.

\Defgeneric {function-inline} {client environment function-name}

This generic function returns the proclaimed inline information of the
function associated with \textit{function-name} in
\textit{environment}.

If \textit{function-name} is not associated with an ordinary function
or a generic function in \textit{environment}, then \texttt{nil} is
returned.

If \textit{function-name} is associated with an ordinary function or a
generic function in \textit{environment}, then the return value of
this function is either \texttt{nil}, \texttt{inline}, or
\texttt{notinline}.  If no inline proclamation has been made, then
this generic function returns \texttt{nil}.

\Defgeneric {(setf function-inline)}\\
{new-inline client environment function-name}

This generic function is used to set the proclaimed inline information
of the function associated with \textit{function-name} in
\textit{environment} to \textit{new-inline}.

\textit{new-inline} must have one of the values \texttt{nil},
\texttt{inline}, or \texttt{notinline}.

If \textit{function-name} is not associated with an ordinary function
or a generic function in \textit{environment}, then an error is
signaled.

\Defgeneric {function-cell} {client environment function-name}

A call to this function always succeeds.  It returns a \texttt{cons}
cell, in which the \texttt{car} always holds the current definition of
the function named \textit{function-name}.  When
\textit{function-name} has no definition as a function, the
\texttt{car} of this cell will contain a function that, when called,
signals an error of type \texttt{undefined-function}.  The return
value of this function is always the same (in the sense of
\texttt{eq}) when it is passed the same (in the sense of
\texttt{equal}) function name and the same (in the sense of
\texttt{eq}) environment.

\Defgeneric {function-unbound} {client environment function-name}

A call to this function always succeeds.  It returns a function that, when
called, signals an error of type \texttt{undefined-function}.  When
\textit{function-name} has no definition as a function, the return value of
this function is the contents of the \texttt{cons} cell returned by
\texttt{function-cell}.  The return value of this function is always the same
(in the sense of \texttt{eq}) when it is passed the same (in the sense of
\texttt{equal}) function name and the same (in the sense of \texttt{eq})
environment.

If the value returned by this function is \texttt{eq} to the \texttt{car} of
the \texttt{cons} cell returned by \texttt{function-cell}, then the function
\textit{function-name} is unbound. The returned function may be called to
signal en error when an attempt is made to evaluate the form
\texttt{(function \textrm{\textit{function-name}})}.

\Defgeneric {function-description} {client environment function-name}

This generic function returns the function description. If no function
description is available, then \texttt{nil} is returned. Result of this
function may be used by the compiler and for introspection.

\Defgeneric {boundp} {client environment symbol}

It returns true if \textit{symbol} has a definition in
\textit{environment} as a constant variable, as a special variable, or
as a symbol macro.  Otherwise, it returns \texttt{nil}.

\Defgeneric {constant-variable} {client environment symbol}

This function returns two values. The first value is true if \textit{symbol}
has a definition as a constant variable and \texttt{nil} otherwise. The second
value is the value of the constant variable \textit{symbol} or \texttt{nil} if
\textit{symbol} doesn't have a definition as a constant variable.

\Defgeneric {(setf constant-variable)} {value client environment symbol}

This function is used in order to define \textit{symbol} as a constant
variable in \textit{environment}, with \textit{value} as its value.

If \textit{symbol} already has a definition as a special variable or
as a symbol macro in \textit{environment}, then an error is signaled.

If \textit{symbol} already has a definition as a constant variable,
and its current value is not \texttt{eql} to \textit{value}, then an
error is signaled.

\Defgeneric {special-variable} {client environment symbol}

This function returns two values. The first value is true if \textit{symbol}
has a definition as a special variable and \texttt{nil} otherwise. The second
value is the value of \textit{symbol} as a special variable in
\textit{environment}. When the special variable is unbound, then the second
value is \texttt{eq} to the result of calling \textit{variable-unbound}.

Notice that the symbol can have a value even though this function returns
\texttt{nil} as its first value.  The first such case is when the symbol has a
value as a constant variable in \textit{environment}.  The second case is when
the symbol was assigned a value using \texttt{(setf symbol-value)} without
declaring the variable as \texttt{special}.

\Defgeneric {(setf special-variable)} {value client environment symbol init-p}

This function is used in order to define \textit{symbol} as a special
variable in \textit{environment}.

If \textit{symbol} already has a definition as a constant variable or
as a symbol macro in \textit{environment}, then an error is signaled.
Otherwise, \textit{symbol} is defined as a special variable in
\textit{environment}.

If \textit{symbol} already has a definition as a special variable, and
\textit{init-p} is \texttt{nil}, then this function has no effect.
The current value is not altered, or if \textit{symbol} is currently
unbound, then it remains unbound.

If \textit{init-p} is true, then \textit{value} becomes the new value
of the special variable \textit{symbol}.

\Defgeneric {symbol-macro} {client environment symbol}

This function returns two values.  The first value is a macro
expansion function associated with the symbol macro named by
\textit{symbol}, or \texttt{nil} if \textit{symbol} does not have a
definition as a symbol macro.  The second value is the form that
\textit{symbol} expands to as a macro, or \texttt{nil} if symbol does
not have a definition as a symbol macro.

It is guaranteed that the same (in the sense of \texttt{eq}) function
is returned by two consecutive calls to this function with the same
(in the sense of \texttt{eq})
symbol as the first argument, as long as the definition of
\textit{symbol} does not change.

\Defgeneric {(setf symbol-macro)} {expansion client environment symbol}

This function is used in order to define \textit{symbol} as a symbol
macro with the given \textit{expansion} in \textit{environment}.

If \textit{symbol} already has a definition as a constant variable, or
as a special variable, then an error of type \texttt{program-error} is
signaled.

\Defgeneric {variable-type} {client environment symbol}

This generic function returns the proclaimed type of the variable
associated with \textit{symbol} in \textit{environment}.

If \textit{symbol} has a definition as a constant variable in
\textit{environment}, then the result of calling \texttt{type-of} on
its value is returned.

If \textit{symbol} does not have a definition as a constant variable
in \textit{environment}, and no previous type proclamation has been
made for \textit{symbol} in \textit{environment}, then this function
returns \texttt{t}.

\Defgeneric {(setf variable-type)} {new-type client environment symbol}

This generic function is used to set the proclaimed type of the
variable associated with \textit{symbol} in \textit{environment}.

If \textit{symbol} has a definition as a constant variable in
\textit{environment}, then an error is signaled.

It is meaningful to set the proclaimed type even if \textit{symbol}
has not previously been defined as a special variable or as a symbol
macro, because it is meaningful to use \texttt{(setf symbol-value)} on
such a symbol.

Recall that the \hs{} defines the meaning of proclaiming the type of a
symbol macro.  Therefore, it is meaningful to call this function when
\textit{symbol} has a definition as a symbol macro in
\textit{environment}.

\Defgeneric {variable-cell} {client environment symbol}

A call to this function always succeeds.  It returns a \texttt{cons}
cell, in which the \texttt{car} always holds the current definition of
the variable named \textit{symbol}.  When \textit{symbol} has no
definition as a variable, the \texttt{car} of this cell will contain
an object that indicates that the variable is unbound.  This object is
the return value of the function \texttt{variable-unbound}.  The
return value of this function is always the same (in the sense of
\texttt{eq}) when it is passed the same symbol and the same
environment.

\Defgeneric {variable-unbound} {client environment symbol}

A call to this function always succeeds.  It returns an object that indicates
that the variable is unbound.  The \texttt{cons} cell returned by the function
\texttt{variable-cell} contains this object whenever the variable named
\textit{symbol} is unbound.  The return value of this function is always the
same (in the sense of \texttt{eq}) when it is passed the same symbol and the
same environment (in the sense of \texttt{eq}).

If the value returned by this function is \texttt{eq} to the \texttt{car} of
the \texttt{cons} cell returned by \texttt{variable-cell}, then the variable
\textit{symbol} is unbound.

\Defgeneric {variable-description} {client environment symbol}

This generic function returns the variable description. If no variable
description is available, then \texttt{nil} is returned. Result of this
function may be used by the compiler and for introspection.

\Defgeneric {find-class} {client environment symbol}

This generic function is a generic version of the Common Lisp function
\texttt{cl:find-class}.

If \textit{symbol} has a definition as a class in
\textit{environment}, then that class metaobject is returned.
Otherwise \texttt{nil} is returned.

\Defgeneric {(setf find-class)} {new-class client environment symbol}

This generic function is a generic version of the Common Lisp function
\texttt{(setf cl:find-class)}.

This function is used in order to associate a class with a class name
in \textit{environment}.

If \textit{new-class} is a class metaobject, then that class
metaobject is associated with the name \textit{symbol} in
\textit{environment}.  If \textit{symbol} already names a class in
\textit{environment} than that association is lost.

If \textit{new-class} is \texttt{nil}, then \textit{symbol} is no
longer associated with a class in \textit{environment}.

If \textit{new-class} is neither a class metaobject nor \texttt{nil},
then an error of type \texttt{type-error} is signaled.

\Defgeneric {class-description} {client environment symbol}

This generic function returns the class description. If no class description
is available, then \texttt{nil} is returned. Result of this function may be
used by the compiler and for introspection.

\Defgeneric {setf-expander} {client environment symbol}

This generic function returns the \texttt{setf} expander associated
with \textit{symbol} in \textit{environment}.  If \textit{symbol} is
not associated with any \texttt{setf} expander in
\textit{environment}, then \texttt{nil} is returned.

\Defgeneric {(setf setf-expander)} {new-expander client environment symbol}

This generic function is used to set the \texttt{setf} expander
associated with \textit{symbol} in \textit{environment}.

If \textit{symbol} is not associated with an ordinary function, a
generic function, or a macro in \textit{environment}, then an error is
signaled.

If there is already a \texttt{setf} expander associated with
\textit{symbol} in \textit{environment}, then the old \texttt{setf}
expander is lost.

If a value of \texttt{nil} is given for \textit{new-expander}, then
any current \texttt{setf} expander associated with \textit{symbol} is
removed.  In this case, no error is signaled, even if \textit{symbol}
is not associated with any ordinary function, generic function, or
macro in \textit{environment}.

\Defgeneric {type-expander} {client environment symbol}

This generic function returns the type expander associated with
\textit{symbol} in \textit{environment}.  If \textit{symbol} is not
associated with any type expander in \textit{environment}, then
\texttt{nil} is returned.

\Defgeneric {(setf type-expander)} {new-expander client environment symbol}

This generic function is used to set the type expander associated with
\textit{symbol} in \textit{environment}.

If there is already a type expander associated with \textit{symbol} in
\textit{environment}, then the old type expander is lost.

\Defgeneric {find-package} {client environment name}

Return the package with the name or the nickname \textit{name} in the
environment \textit{environment}.  If there is no package with that
name in \textit{environment}, then return \texttt{nil}.  Contrary to
the standard \commonlisp{} function \texttt{cl:find-package}, for this
function, \textit{name} must be a string.

\Defgeneric {(setf find-package)} {new-package client environment name}

This function is used in order to associate a package with a package
name in \textit{environment}.  The argument \textit{name} must be a
string.

If \textit{new-package} is a package object, then that package object
is associated with the name \textit{name} in
\textit{environment}.  If \textit{name} already names a package in
\textit{environment} than that association is lost.

If \textit{new-package} is \texttt{nil}, then \textit{name} is no
longer associated with a package in \textit{environment}.

If \textit{new-package} is neither a package object nor \texttt{nil},
then an error of type \texttt{type-error} is signaled.
