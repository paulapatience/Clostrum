\chapter{Using \trucler{} with \sysname{}}

\section{ASDF system definition and package}

\trucler{} can optionally be used with \sysname{}.  The ASDF system
\texttt{clostrum-trucler} should be used for this purpose.

The code for the method definitions described in this chapter is in a
separate package named \texttt{clostrum-trucler}.

\section{\trucler{} methods}

\defmethod {describe-variable} {client (environment
  {\tt run-time-environment}) \\ name}
\Defmethod {describe-variable} {client (environment
  {\tt compilation-environment}) \\ name}

These methods make an instance of an appropriate \trucler{} class
as follows:

\begin{itemize}
\item If \textit{name} is the name of a special variable in
  \textit{environment}, then an instance of the \trucler{} class
  \texttt{global-special-variable-description} is made.
\item If \textit{name} is the name of a constant variable in
  \textit{environment}, then an instance of the \trucler{} class
  \texttt{constant-variable-description} is made.
\item If \textit{name} is the name of a symbol macro in
  \textit{environment}, then an instance of the \trucler{} class
  \texttt{global-symbol-macro-description} is made.
\end{itemize}

\defmethod {describe-function} {client (environment
  {\tt run-time-environment}) \\ name}
\Defmethod {describe-function} {client (environment
  {\tt compilation-environment}) \\ name}

These methods make an instance of an appropriate \trucler{} class
as follows:

\begin{itemize}
\item If \textit{name} is the name of an ordinary function in
  \textit{environment}, then an instance of the \trucler{} class
  \texttt{global-function-description} is made.
\item If \textit{name} is the name of a generic function in
  \textit{environment}, then an instance of the \trucler{} class
  \texttt{generic-function-description} is made.
\item If \textit{name} is the name of a macro in
  \textit{environment}, then an instance of the \trucler{} class
  \texttt{global-macro-description} is made.
\item If \textit{name} is the name of a special operator in
  \textit{environment}, then an instance of the \trucler{} class
  \texttt{special-operator-description} is made.
\end{itemize}

\Defmethod {describe-block} {client (environment
  {\tt compilation-environment}) \\ name}

This method returns \texttt{nil}.

\Defmethod {describe-tag} {client (environment
  {\tt compilation-environment}) \\ name}

This method returns \texttt{nil}.

\defmethod {global-environment} {client (environment
  {\tt run-time-environment})}
\Defmethod {global-environment} {client (environment
  {\tt compilation-environment})}

These methods return \textit{environment}.
