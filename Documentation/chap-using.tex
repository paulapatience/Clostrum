\chapter{Using \sysname{}}

There are several use cases for \sysname{} depending on the type and
level of ambition of the client.

\section{Implementing standard environment functions}

Unfortunately, the \commonlisp{} standard does not contain a separate
chapter dedicated environment functions.%
\footnote{The chapter of the standard
  entitled ``Environment'' has to do with the computational environment
  of the \commonlisp{} system.}
So for example the function \texttt{symbol-function} is specified in
the chapter entitled ``Symbols'', whereas the function
\texttt{fdefinition} is specified in the chapter entitled ``Data and
Control Flow'', even though the two are very much related.

In this section, we describe how a \commonlisp{} implementation could
implement the standard environment functions using a \sysname{}
run-time environment as a first-class global environment.  To access
the \sysname{} run-time environment, we are going to assume it is
available in the host special variable \texttt{*e*}, but in
a real system, accessing the value of a special variable may require
access to the global environment, so some other mechanism might be
needed.  Similarly, we assume that some kind of client object is
available in the special variable \texttt{*c*}.

We do not include code for error checking in our examples.  In many
cases, the host \commonlisp{} system may choose to include or exclude
error checking depending on the value of the \texttt{safety} optimize
quality.  In our examples, we use the package prefix \texttt{env:} to
refer to \sysname{} symbols.

\subsection{\texttt{fboundp}}

The standard says that \texttt{fboundp} returns a true value if and
only if the name is defined either as a function, a macro, or a
special operator in the global environment.  We use the \sysname{}
functions \texttt{fdefinition}, \texttt{macro-function}, and
\texttt{special-operator} to query the environment.  Each of these
functions returns \texttt{nil} if no definition is present, so if all
of them return \texttt{nil}, then the name is not fbound.  The
standard says that, in safe code, the implementation must check that
the name is a valid function name, and we indicate the place for that
verification in a comment.

\begin{verbatim}
(defun fboundp (name)
  ;; At least in safe code, check that NAME is a valid
  ;; function name.
  (not (and (null (env:fdefinition *c* *e* name)))
            (null (env:macro-function *c* *e* name))
            (null (env:special-operator *c* *e* name))))
\end{verbatim}

\subsection{\texttt{fdefinition} and \texttt{symbol-function}}

The standard leaves several decisions about the behavior of
\texttt{fdefinition} up to the implementation.  It is specified that a
function object is returned if the name is defined as a function.  If
the name is defined as a macro or a special operator, the standard
specified that a true value must be returned, but it leaves the nature
of the value up to the implementation.  In our example implementation,
we assume that the \commonlisp{} implementation calls the \sysname{}
function \texttt{(setf special-operator)} with a value that is
acceptable as a return value for the standard function
\texttt{fdefinition} when the name is that of a special operator.

\begin{verbatim}
(defun fdefinition (name)
  ;; At least in safe code, check that NAME is a valid
  ;; function name.
  (let ((function (env:fdefinition *c* *e* name)))
    (if (not (null function))
        function
        (let ((macro (env:macro-function *c* *e* name)))
          (if (not (null macro))
              `(:macro ,macro)
              (let ((special (env:special-operator *c* *e* name)))
                (if (not (null special))
                    special
                    (error 'undefined-function ...))))))))
\end{verbatim}

The standard function \texttt{symbol-function} is like
\texttt{fdefinition} except that the name is restricted to be a
symbol, which the implementation must verify in safe code.

\begin{verbatim}
(defun symbol-function (name)
  ;; At least in safe code, check that NAME is a symbol
  (fdefinition name))
\end{verbatim}

\section{Compilation environment only}

An existing (perhaps relatively new) \commonlisp{} implementation that
has decided to take advantage of the permission to merge the run-time
environment and the evaluation environment, but that wants a separate
compilation environment can use \sysname{} as follows:

\begin{itemize}
\item It would define an environment class  (say
  \texttt{client:global-environment}) that has
  \texttt{clostrum:run-time-environment} as a superclass.
\item It would implement methods on the generic functions in
  \refSec{sec-run-time-protocol-functions} that trampoline to existing
  implementation-specific functions in the single global environment
  of the client.
\item It would create a constellation consisting of an instance of
  \texttt{client:global-environment} as the parent of an instance of
  \texttt{clostrum:compilation-environment}
\end{itemize}

\section{Compilation and evaluation environment}

An existing \commonlisp{} implementation with a traditional, single
global run-time environment that wants a separate evaluation
environment in order to avoid compile-time side effects to its
run-time environment can use \sysname{} as follows:

\begin{itemize}
\item It would define an environment class  (say
  \texttt{client:run-time-environment}) that has
  \texttt{clostrum:environment} as a superclass.
\item It would implement methods on the generic functions in
  \refSec{sec-run-time-protocol-functions} that trampoline to existing
  implementation-specific functions in the single global environment
  of the client.
\item It would create a class (say
  \texttt{client:evaluation-environment}) as a subclass of
  \texttt{clostrum:evaluation-environment-mixin} and
  \texttt{client:run-time-environment}.
\item It would create a constellation consisting of an instance of
  \texttt{client:run-time-environment}, an instance of
  \texttt{client:evaluation-environment},
  \texttt{clostrum:compilation-environment}
\end{itemize}
